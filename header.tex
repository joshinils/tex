\usepackage{amsmath,amssymb, a4, verbatim}
\usepackage[german]{babel}
%\usepackage[latin1]{inputenc}
\usepackage[utf8]{inputenc} % ueoeaess
\usepackage{listings} % fuer inline codelistings
\lstset{%
    basicstyle=\ttfamily,    % the size of the fonts
    columns=fixed,            % anything else is horrifying
    showspaces=false,        % show spaces using underscores?
    showstringspaces=false,    % underline spaces within strings?
    showtabs=false,            % show tabs within strings?
    xleftmargin=1.5em,        % left margin space
}
\lstdefinestyle{inline}{basicstyle=\ttfamily}
\newcommand{\listline}[1]{\lstinline[style=inline]!#1!}

\usepackage{caption}
\newcommand{\tinycaption}[1]{\captionsetup{labelformat=empty}\caption{#1}}
\captionsetup{belowskip=-10pt, aboveskip=5pt}

%\usepackage{color}
%\usepackage{epsfig} % eps
\usepackage{graphicx} % eps
%\usepackage[shortcuts]{extdash}
%\usepackage{dsfont}
%\usepackage{epstopdf} % eps
%\usepackage[pdf]{pstricks} % eps
%\usepackage{auto-pst-pdf}
\usepackage{mathtools}
\usepackage{dsfont} % $ \mathds{1} $
\usepackage{icomma}
\usepackage{tikz}
%\usepackage{pgfplots}
%\pgfplotsset{compat=1.8}
\usepackage[bottom]{footmisc} % put footnotes at the bottom of page
\usepackage{nicefrac} % fuer brueche die aussehen wie prozentzeichen
% \usepackage{ps2pdf}
\usetikzlibrary{automata,positioning}

\usepackage{algorithmicx}
\usepackage{algpseudocode}
\usepackage{algorithm}

\usepackage{multicol}
\usepackage{wrapfig} % make stuff float
\usepackage{placeins} % stop stuff from floating
\usepackage{seqsplit} % very long numbers
\usepackage{framed} % begin{framed}

%  Headings and Footings :
\usepackage{fancyhdr}
\headheight20pt
\lhead{\lheadcontent}

\chead{}
\rhead{\thepage}
\renewcommand{\headrulewidth}{.4pt}
\lfoot{\today}
\cfoot{}
\rfoot{\input{rfoot.tex}}
\renewcommand{\footrulewidth}{.4pt}

\usepackage[all=tight, charwidths=normal, wordspacing=normal, mathspacing=normal]{savetrees}

\usepackage{titlesec}
\titlespacing*{\section      }{0pt}{  2ex plus 0ex minus 0ex}{0ex plus 0ex}
\titlespacing*{\subsection   }{0pt}{1.5ex plus 0ex minus 0ex}{0ex plus 0ex}
\titlespacing*{\subsubsection}{0pt}{  1ex plus 0ex minus 0ex}{0ex plus 0ex}


\usepackage{enumitem}
\setitemize{itemsep=5pt,topsep=0pt,parsep=0pt,partopsep=0pt}

\textwidth16.5cm
\oddsidemargin0.cm
\evensidemargin0.cm

\headsep20pt

\usepackage{calc}
\newlength{\afourwidth}
\setlength{\afourwidth}{210mm}

\renewcommand{\baselinestretch}{1}

%disallow hyphenation:
\hyphenpenalty=1000
\exhyphenpenalty=1500

\usepackage[]{geometry}
\geometry{
    a4paper, %landscape,
    %paperheight=297mm, paperwidth=210mm,
    paperheight=297mm, paperwidth=\afourwidth*\real{\AfourWidthMult}, % multicols(2) but one one page width
    left=\somedistanceLeft,
    right=\somedistanceLeft,
    top=\somedistanceTop,
    bottom=\somedistanceTop
}

\parindent0pt
\usepackage{parskip}

\newcommand{\R}{ {\mathbb R} }
\newcommand{\Z}{ {\mathbb Z} }
\newcommand{\C}{ {\mathbb C} }
\newcommand{\e}{ {\,\mathrm e\,} }
\newcommand{\imag}{ {\mathrm i} }
\newcommand{\jmag}{ {\mathrm j} }
\newcommand{\1}{ {\mathds{1}} }
\newcommand{\abs}[1]{\lvert#1\rvert}
\newcommand{\norm}[1]{\left\lVert#1\right\rVert}
\newcommand{\xt}{\tilde{x}}
\newcommand{\dotleq}{\dot{\leq}}
\newcommand{\m}{\hphantom{-} }
\newcommand{\grad}[1]{\mathop{\text{grad}}\nolimits#1}
\newcommand{\dashfill}[1]{\vspace{11pt}\def\dashfill{\cleaders\hbox{#1}\hfill}\hbox to \hsize{\dashfill\hfil}\vspace{11pt}}
\newcommand{\scdot}{\!\cdot\!}
\newcommand{\sig}{\text{signum}}
\newcommand{\rot}{}
\newcommand{\Jacobi}[1]{\mathop{\mathrm{J}}\nolimits#1}
\newcommand{\laplace}{\mathop{{}\Delta}\nolimits}
\newcommand{\td}{\,\text{d}}
\newcommand{\kommentar}[1]{}
\newcommand{\Real}[1]{\mathop{{}\mathrm{Re}}\left\{#1\right\}}

% https://tex.stackexchange.com/questions/37080/how-can-i-indent-a-block-of-text-for-a-specified-amount/37084
\usepackage{scrextend}

% https://tex.stackexchange.com/questions/256894/how-to-make-underbrace-label-text-wrap-at-brace-width
\def\underbraceWrap#1_#2{%
    \setbox0=\hbox{$\displaystyle#1$}%
    \underbrace{\centering #1}_{\parbox[t]{\the\wd0}{\centering{#2}}}%
}

\everydisplay{\ifdim\predisplaysize=-\maxdimen \predisplaysize=\hsize \fi}
\usepackage{nccmath}
\usepackage{etoolbox}

% https://tex.stackexchange.com/questions/238505/change-size-of-font-of-subscript-of-underbrace-in-align-environment/238510#238510
\usepackage{mathtools}
\usepackage{relsize}

% https://tex.stackexchange.com/questions/80378/red-squiggly-imitation
\usepackage[normalem]{ulem} % [normalem] to set \emph{} to default behaviour (italic instead of underlining)

\newcommand{\mcomm}[1][1fil]{\hskip 1em plus #1 minus 1em \quad\vert\; }

\usepackage{setspace}

% https://tex.stackexchange.com/questions/57338/how-can-i-allow-gather-to-break-between-pages
\allowdisplaybreaks

%\renewcommand{\theequation}{\arabic{section}.\arabic{equation}}
%\numberwithin{equation}{section}

% https://tex.stackexchange.com/questions/264394/equation-number-syntax-section-subsection-subsubsection-equationnu
% \renewcommand{\theequation}{%
% \ifnum\value{subsection} > 0
% \ifnum\value{subsubsection} > 0
% \thesubsubsection--\arabic{equation}%
% \else
% \thesubsection--\arabic{equation}%
% \fi
% \else
% \thesection--\arabic{equation}%
% \fi
% }
\renewcommand{\theequation}{%
\ifnum\value{subsection} > 0
\ifnum\value{subsubsection} > 0
\thesubsubsection--\arabic{equation}%
\else
\thesubsection--\arabic{equation}%
\fi
\else
\thesection--\arabic{equation}%
\fi
}

% https://tex.stackexchange.com/questions/207532/reset-equation-numbering-after-each-section
\usepackage{chngcntr}
\counterwithin*{equation}{section}
\counterwithin*{equation}{subsection}
\counterwithin*{equation}{subsubsection}

% define text and background-color
\usepackage{xcolor}
\definecolor{bgcolor}{RGB}{33, 33, 33}
\pagecolor{bgcolor}
\definecolor{txcolor}{RGB}{220, 220, 220}
\makeatletter
\newcommand{\globalcolor}[1]{%
  \color{#1}\global\let\default@color\current@color
}
\makeatother
\AtBeginDocument{\globalcolor{white}}

% https://tex.stackexchange.com/questions/42507/wrapfigure-cuts-off-images-at-the-end-of-the-page
% positions wrapfigure not too low on page
\makeatletter
\newcommand{\checkheight}[1]{%
  \par \penalty-100\begingroup%
  \setbox8=\hbox{#1}%
  \setlength{\dimen@}{\ht8}%
  \dimen@ii\pagegoal \advance\dimen@ii-\pagetotal
  \ifdim \dimen@>\dimen@ii
    \break
  \fi\endgroup}
\makeatother

\usepackage{blindtext}

% https://tex.stackexchange.com/questions/518936/using-computer-modern-int-symbol-with-txfonts-math-font
% https://github.com/joshinils/tex/blob/master/greek.tex
\usepackage{newtxtext}% For text font
\usepackage{newtxmath}

\usepackage{upgreek}

\renewcommand{\alpha  }{\mathit{\alphaup}   }
\renewcommand{\epsilon}{\varepsilon}
\renewcommand{\phi    }{\varphi    }
\renewcommand{\theta  }{\upvartheta}
\renewcommand{\gamma  }{\gammaup   }
\renewcommand{\sigma  }{\sigmaup   }
\renewcommand{\lambda }{\lambdaup  }

\DeclareSymbolFont{largesymbolscmr}{OMX}{cmex}{m}{n}
\DeclareMathSymbol{\intop}{\mathop}{largesymbolscmr}{"52}
\DeclareRobustCommand\int{\intop\nolimits}

% https://tex.stackexchange.com/questions/3622/best-way-to-generate-nice-function-plots-in-latex
\usetikzlibrary{datavisualization}
\usetikzlibrary{datavisualization.formats.functions}

\DeclareSymbolFont{largeesint}{U}{esint}{m}{n}
\DeclareMathSymbol{\cmriintop}{\mathop}{largeesint}{"03}
\DeclareRobustCommand\iint{\cmriintop\nolimits}

\usepackage{fonttable}

\usepackage[range-phrase=\text{ bis },exponent-product=ep789, inter-unit-product=iup456, output-decimal-marker=odm123]{siunitx}
\sisetup{locale = DE}

% makes nested parentheses bigger
\delimitershortfall=0pt

\usepackage[hyphens]{url}
\usepackage{hyperref}

